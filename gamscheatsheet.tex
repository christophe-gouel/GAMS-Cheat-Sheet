\documentclass[10pt,landscape,a4paper]{article}

\usepackage{array,tabularx,booktabs,microtype}

\usepackage{graphicx}

% Color
\usepackage{color}
\definecolor{DarkBlue}{rgb}{0,0.08,0.45}

\usepackage{multicol}
\setlength{\columnseprule}{1pt}\usepackage{calc}
\usepackage{ifthen}
\usepackage[landscape]{geometry}
\usepackage{hyperref}
\hypersetup{%
  bookmarksopen=false,%
  citecolor=black,%
  urlcolor=DarkBlue,%
  linkcolor=black,%
  filecolor=black,%
  colorlinks=true,%
  pdfauthor={Christophe Gouel},%
  pdftitle={GAMS Cheat Sheet},%
  pdfstartview={Fit}}

\usepackage{etoolbox}
\makeatletter
\preto{\@verbatim}{\topsep=0pt \partopsep=0pt }
\makeatother

% This sets page margins to .5 inch if using letter paper, and to 1cm
% if using A4 paper. (This probably isn't strictly necessary.)
% If using another size paper, use default 1cm margins.
\ifthenelse{\lengthtest { \paperwidth = 11in}}
	{ \geometry{top=.5in,left=.5in,right=.5in,bottom=.5in} }
	{\ifthenelse{ \lengthtest{ \paperwidth = 297mm}}
		{\geometry{top=1cm,left=1cm,right=1cm,bottom=1cm} }
		{\geometry{top=1cm,left=1cm,right=1cm,bottom=1cm} }
	}

% Turn off header and footer
\pagestyle{empty}

% Redefine section commands to use less space
\makeatletter
\renewcommand{\section}{\@startsection{section}{1}{0mm}%
                                {-1ex plus -.5ex minus -.2ex}%
                                {0.5ex plus .2ex}%x
                                {\color{blue}\normalfont\large\bfseries}}
\renewcommand{\subsection}{\@startsection{subsection}{2}{0mm}%
                                {-1explus -.5ex minus -.2ex}%
                                {0.5ex plus .2ex}%
                                {\normalfont\normalsize\bfseries}}
\renewcommand{\subsubsection}{\@startsection{subsubsection}{3}{0mm}%
                                {-1ex plus -.5ex minus -.2ex}%
                                {1ex plus .2ex}%
                                {\normalfont\small\bfseries}}
\makeatother

% Define BibTeX command
\def\BibTeX{{\rm B\kern-.05em{\sc i\kern-.025em b}\kern-.08em
    T\kern-.1667em\lower.7ex\hbox{E}\kern-.125emX}}

% Don't print section numbers
\setcounter{secnumdepth}{0}


\setlength{\parindent}{0pt}
%\setlength{\parskip}{0pt plus 0.5ex}
\setlength{\parskip}{1pt plus 0.5ex}

% -----------------------------------------------------------------------

\begin{document}

\raggedright
\footnotesize
\begin{multicols}{2}


% multicol parameters
% These lengths are set only within the two main columns
%\setlength{\columnseprule}{0.25pt}
\setlength{\premulticols}{1pt}
\setlength{\postmulticols}{1pt}
\setlength{\multicolsep}{1pt}
\setlength{\columnsep}{2pt}

\begin{center}
     \Large{\textbf{GAMS Cheat Sheet}} \\
\end{center}

\section{Declarations}
GAMS objects have to be declared before their first use. Main objects are\\
\begin{tabularx}{\columnwidth}{@{}>{\ttfamily}l>{\raggedright\arraybackslash}X@{}}
  Set & Collection of elements used for indexing. $S=\left\{a,b,c\right\}$ is
  written in GAMS as \texttt{Set S / a, b, c /;}. A sequence of elements, such
  as t=1990:2010, can be entered as \texttt{Set t "Year" / 1900*2010 /;}, where
  \texttt{"Year"} is an optional
  explanatory text.\\
  Parameter\textrm{, }Scalar\textrm{, }Table & Exogenous data (given input) to
  be entered or calculated by the modeler. \texttt{Scalar} is 0-dimenional,
  \texttt{Parameter} is $n$-dimensional and \texttt{Table} is a $n$-dimensional parameter
  that expects the input in table format.
  \\
  Variable & Endogenous variables to be determined by GAMS. It is possible to
  enter the following prefixes before \texttt{Variable} to specify the variable
  type:
  \texttt{Positive}, \texttt{Negative}, \texttt{Binary} (variable is 0 or 1), \texttt{Integer}.\\
  Equation   & Keyword to define an algebraic relationship between variables. \\
  Model & Collection of equations. To declare a model that includes all the
  equations:\linebreak{} \texttt{Model} \emph{model\_name} \texttt{/ all
    /;}. \linebreak{}To include a list of equations:\linebreak{} \texttt{Model}
  \emph{model\_name} \texttt{/} \emph{eq\_name1}\texttt{,} \emph{eq\_name2}
  \texttt{/;}
\end{tabularx}

\section{Data entry}
The general expression to declare and define data is\\
\emph{data\_type} \emph{symbol\_name} [\emph{symbol\_description}] [\verb!/! \emph{symbol\_value} \verb!/!]\verb!;!\\
Examples:
\begin{verbatim}
Scalar rho "discount rate" / .15 /;
Parameters b(i) / seattle 20, san-diego 45 /
           salaries(employee,manager,department)
             / anderson .murphy .toy       = 6000
                 hendry .smith  .toy       = 9000
                hoffman .morgan .cosmetics = 8000 /;
\end{verbatim}

\section{Variable attributes}
To each variable is associated a series of attributes:\\
\begin{tabular}{@{}>{\ttfamily}ll@{}}
  .l & Level of the variable. Receives new values when a model is solved.\\
  .lo & Lower bound (default to \texttt{-inf}).\\
  .up & Upper bound (default to \texttt{inf}).\\
  .fx & To fix a variable (sets the value for the attributes \texttt{.l}, \texttt{.lo} and \texttt{.up}):\linebreak{}
  \texttt{x.fx(i) = 1;}\\
  .m & Marginal (or dual) value. Receives new values when a model is solved.
\end{tabular}

\section{Arithmetic and functions}
Arithmetic operations:\\
\verb!+!, \verb!-!, \verb!*!, \verb!/!, \verb!**! (exponentiation, \verb!x**y!
is defined only for \verb!x!$\ge$0, if \texttt{x} can be negative, use
\verb!power(x, y)! instead).

Most common functions
(see the \href{https://www.gams.com/latest/docs/UG_Parameters.html#UG_Parameters_IntrinsicFunctions}{documentation} for the list of intrinsic functions):\\
\verb!abs()!, \verb!cos()!, \verb!exp()!, \verb!log()!, \verb!log10()!,
\verb!max(,...,)!, \verb!min(,...,)!, \verb!power(,)!, \verb!round()!,
\verb!sin()!.

Relationship operators:\\
\verb!lt!, \verb!<!, \verb!le!, \verb!<=!, \verb!eq!, \verb!=!, \verb!ne!, \verb!<>!, \verb!ge!, \verb!>=!, \verb!gt!, \verb!>!.

Logical operators:\\
\verb!not!, \verb!and!, \verb!or!, \verb!xor!.

Special symbols:\\
\begin{tabularx}{\columnwidth}{@{}>{\ttfamily}l>{\raggedright\arraybackslash}X@{}}
  inf& Plus infinity.\\
  -inf& Minus infinity.\\
  na& Not available, used for missing data.\\
  undf& Undefined, result of an undefined operation such as \texttt{3 / 0}.\\
  eps& Numerically equal to zero, but considered as existing. For example,
  \texttt{sum(i \$ z(i), 1)} equals \texttt{0} if \texttt{z(i) = 0} and
  \texttt{sum(i \$ z(i), 1)} equals \texttt{card(i)} if \texttt{z(i) = eps}.
\end{tabularx}

\section{Conditional expressions with dollar condition}

Logical expression can be expressed with a dollar condition. For example:\\
\texttt{a \$ (b > 1.5) = 2;} means if \texttt{b} is greater than \texttt{1.5} then
\texttt{a} equals \texttt{2}. If \texttt{b} is less than-or-equal to
\texttt{1.5} then the value of \texttt{a} remains unchanged.

It can also be used on the right hand side. For example:\\
\texttt{a = 2 \$ (b > 1.5);} means that \texttt{a} equals \texttt{2} if \texttt{b}
is greater than \texttt{1.5}, else \texttt{a} equals \texttt{0}.

\section{Indexing}

\subsection{Basic indexing}

\begin{tabularx}{\columnwidth}{@{}>{\ttfamily}lX@{}}
x(i) = 12; & Assign all elements of \texttt{x} to 12.\\
b("seattle") = 20; & Assign the element \texttt{seattle} of \texttt{b} to 20.\\
sum(i, x(i)) & Sum \texttt{x} over the set \texttt{i}: $\sum_{i}x_{i}$.\\
sum((i,j), x(i,j)) & Sum \texttt{x} over the sets \texttt{i} and \texttt{j}: $\sum_{i,j}x_{i,j}$.\\
prod(j, y(i,j) & Multiply \texttt{y} over the set \texttt{j}:
$\prod_{j}y_{i,j}$.\\
Alias (i,j) & Declare that the set \texttt{j} can be used instead of
\texttt{i}.\\
y = smax(i, x(i)); \textrm{or} y = smin(i, x(i)); & Find the largest or smallest
value of a symbol indexed over a set.
\end{tabularx}

\subsection{Advanced indexing}

On ordered sets (for example one defined by \texttt{Set t "Year" / 1900*2010 /;}):
\begin{tabularx}{\columnwidth}{@{}>{\ttfamily}l>{\raggedright\arraybackslash}X@{}}
ord(t)& Returns the position of a member in a set:\linebreak{}
\texttt{Parameter val(t);}\linebreak{}
\texttt{val(t) = ord(t);}\linebreak{}
Here \texttt{val("1900")} will be \texttt{1}, \texttt{val("1909")} \texttt{10}, and \texttt{val("2010")} \texttt{111}.\\
card(t) & Returns the number of elements in a set: \texttt{card(t)} will return \texttt{111}.\\
\textrm{lags and leads} & It is possible to use lag or lead operators on ordered sets. For example an equation defining the evolution of capital stock would
be:\linebreak{}
\texttt{eq\_k(t+1) .. k(t+1) =e= (1 - delta) * k(t) + i(t);}
\end{tabularx}

\texttt{sameAs(r,s)} can be used to test if the active elements of \texttt{r} and \texttt{s} are the same. For example: \texttt{a(r,s) \$ (not sameAs(r,s)) = 10;} would assign \texttt{10} to all non-diagonal elements of \texttt{a}.

It is possible to define subsets: sets whose members must all be members of some
larger sets. For example:\\
\begin{verbatim}
Sets
  i "all sectors" / light-ind, food+agr, heavy-ind, services /
  t(i) "traded sectors" / light-ind, food+agr, heavy-ind /;
\end{verbatim}
Note that a subset is ordered when the indices are entered in the same order as
the ordered parent set.

The assignment can also be made dynamically (the set is then called a dynamic
set):
\begin{tabularx}{\columnwidth}{@{}>{\ttfamily}l>{\raggedright\arraybackslash}X@{}}
Set j(i);& Declare \texttt{j} as a subset of \texttt{i}. \\
j(i) = yes;& Assign all elements of \texttt{i} to \texttt{j}.\\
j("light-ind") = no;& Remove the element \texttt{"light-ind"} from \texttt{j}.
\end{tabularx}
Or alternatively: \texttt{j(i) \$ (not sameAs(i,"light-ind")) = yes;}.

Dynamic subsets present the following restrictions: it is not possible to
declare variables defined on dynamic subsets; and they are not ordered, even if
their parent sets are.

\section{Equation definition}
An equation named \emph{eqname} is defined by\\
\emph{eqname}(\emph{index}) \verb!..! expression \emph{eq\_type} expression \verb!;!\\
Example: \verb!cost .. z =e= sum((i,j), c(i,j) * x(i,j));!

Main equation types (\emph{eq\_type}):\\
\begin{tabular}{@{}ll@{}}
  \verb!=e=! & Equality: rhs must equal lhs.\\
  \verb!=g=! & Greater than: lhs must be greater than or equal to rhs.\\
  \verb!=l=! & Less than: lhs must be less than or equal to rhs.
\end{tabular}

\columnbreak{}
\section{Solve statement}
\verb!solve! \emph{model\_name} \verb!using! \emph{model\_type} (\verb!maximizing|minimizing! \emph{objective\_name})\verb!;!\\
Example: \verb!solve transport using lp minimizing z;!

Main model types [\emph{model\_type}] ((see the
\href{https://www.gams.com/latest/docs/UG_ModelSolve.html#UG_ModelSolve_ModelClassificationOfModels}{documentation}
for the complete list):\\
\begin{tabularx}{\columnwidth}{@{}>{\ttfamily}lX@{}}
cns  & Constrained Nonlinear System: square system of nonlinear
equations, $f\left(x\right)=0$.\\
lp  & Linear programming: optimization problem with linear objective and constraints.\\
mcp  & Mixed complementarity problem.\\
mip & Mixed integer programming: linear opt. pb. with a mix of
continuous and integer variables.\\
nlp  & Nonlinear programming: optimization problem with nonlinear
objective and constraints.\\
qcp  & Quadratic constraint programming: optimization problem with
quadratic objective and constraints.
\end{tabularx}

\section{Display}

\verb!display x, y.l;! to ask GAMS to write in the listing file (file with the
\texttt{.lst} extension) the value of \verb!x!  and \verb!y!. For variables, one
has to precise the attribute (\verb!.l! here).

\verb!option decimals = N! to restrict the display to the first \verb!N!
decimals.

\section{Flow control}

GAMS contains 4 types of loops (use \texttt{break} and \texttt{continue} to break out of a loop and to jump to the next iteration):
\begin{description}
\item[\texttt{for}] to loop over a parameter:\\
\begin{multicols}{2}
\begin{verbatim}
Scalar i;
for (i = 1 to 1000 by 10,
  display i;
);
\end{verbatim}
  \columnbreak
\begin{verbatim}

for (i = 1000 downto 1 by 10,
  display i;
);
\end{verbatim}
\end{multicols}
\item[\texttt{loop}] to loop over a set:\\
  \verb!loop (t, pop(t+1) = pop(t) + growth(t));!
\item[\texttt{while}] to loop over a logical condition:\\
\begin{verbatim}
Scalar x / 0 /;
while (x <= 10, x = x + 1;);
\end{verbatim}
\item[repeat] to loop over a logical condition with at least one execution
\begin{verbatim}
repeat (x = x + 1; until x <= 10);

\end{verbatim}
\end{description}

Use of the \verb!if-else! statement:\\
\begin{verbatim}
if (x <= 0,               y = 1;
elseif (x > 0 and x < 1), y = 2;
else                      y = 3;
);
\end{verbatim}

To stop GAMS if a condition is met use \verb!abort!:\\
\verb!abort $ (abs(residuals) > 1E-6) "Residual not null", residuals;!

\section{Dollar control}

Dollar control options can alter GAMS behavior in several ways. The \texttt{\$}
symbol can be placed in the first column or elsewhere on a line if using
\texttt{\$\$} as first two characters. They are executed at compile time, so
before any calculation takes place. Most important dollar control options (see
the
\href{https://www.gams.com/latest/docs/UG_DollarControlOptions.html}{documentation}
for the complete list):
\begin{tabularx}{\columnwidth}{@{}>{\ttfamily}lX@{}}
\$exit & GAMS stop reading the file after \texttt{\$exit}.\\
\$include & Use \texttt{\$include filename} to insert the contents of the
file.\\
\$onText\textrm{/}\$offText & Use to enclose severals lines of comments.\\
\$set & Use \texttt{\$set varname varvalue} to define an environmental
variable (also called control variable), which value can be accessed later by using
\texttt{\%varname\%}. Additionally, you can set a control variable via a user defined
command line parameter option, e.g., \texttt{gams~trnsport.gms --varname=varvalue}.
\end{tabularx}

\section{Options}

Some options can be set using the following syntax:\\
\texttt{option} \emph{option\_name} \texttt{=} \emph{option\_value}\texttt{;}\\
Main options (see the
\href{https://www.gams.com/latest/docs/UG_OptionStatement.html}{documentation}
for the complete list):\\
\begin{tabularx}{\columnwidth}{@{}>{\ttfamily}lcX@{}}
  \emph{\textrm{option\_name}} & Default & Interpretation \\
  \toprule
  decimals & 3 & Number of decimals printed.\\
  iterLim& 2e9 & Limit on the number of iterations used to solve a model. \\
  limCol& 3 & Control the number of columns (variables) listed at each solve.\\
  limRow& 3 & Control the number of rows (equations) listed at each solve.\\
  resLim& 1000 & Limit on the units of processor time used to solve a model.\\
  \emph{\textrm{solver}} \textrm{(}cns\textrm{, }nlp\textrm{, }lp\textrm{, \ldots)}&
  Installation default& Control the solver used to solve a particular model type.
\end{tabularx}
Example: \texttt{option limCol = 0;}

\section{Comments}

A line starting with an asterisk '\verb!*!' is commented:\\
\verb!* This line is a comment!

To comment several lines, it is possible to place them between a pair of
\verb!$onText!/\verb!$offText!:\\
\begin{verbatim}
$onText
Any lines between $onText and $offText are commented
$offText
\end{verbatim}

End-of-line comments can be enabled using \verb!$eolCom! followed by the
chosen special character:\\
\begin{verbatim}
$eolCom #
x = 1; # This is an end-of-line comment
\end{verbatim}

In-line comments can be enabled using \verb!$inlineCom! followed by a pair of
one or two character sequence (default to \verb!/* */!):\\
\begin{verbatim}
$inlineCom { }
x { This is an in-line comment } = 1;
\end{verbatim}

\section{GDX files}
\label{sec:gdx-files}

A GDX file is a binary file that can contains information on sets, parameters,
variables, and equations. GDX files are very useful to enter data, to explore
results, and to import/export data from various file formats (e.g., csv, Excel,
\ldots).

\subsection{Compile phase (before any calculation)}

\begin{tabularx}{1.0\columnwidth}{@{}>{\ttfamily}lX@{}}
  \$gdxIn \emph{\textrm{file\_name.gdx}} & Open the GDX file for reading. \\
  \$load \emph{\textrm{id1 id2=gdxid2}} & Read symbols \emph{id1} and
  \emph{gdxid2} from the GDX
  file and assign them to \emph{id1} and \emph{id2} that have been previously declared. \\
  \$gdxIn & Close the GDX file currently open.
\end{tabularx}
Same thing with \texttt{\$gdxOut} and \texttt{\$unLoad} to write data to a GDX
file during the compile phase.

\subsection{Execution phase (after calculations)}

\begin{tabularx}{1.0\columnwidth}{@{}>{\ttfamily}lX@{}}
  execute\_load "\emph{\textrm{file\_name.gdx}}" \emph{\textrm{id1, id2=gdxid2}}
  & Read symbols \emph{id1} and \emph{gdxid2} from the GDX
  file and assign them to \emph{id1} and \emph{id2} that have been previously declared.\\
  execute\_unload "\emph{\textrm{file\_name.gdx}}" \emph{\textrm{id1,
      id2=gdxid2}} & Write to the GDX file the symbols \emph{id1} and \emph{id2}
  and assign \emph{id2} to the symbol \emph{gdxid2}.
\end{tabularx}

\vspace{0.5cm}
\rule{3cm}{0.5pt}

\href{http://creativecommons.org/publicdomain/zero/1.0/}{\includegraphics[height=1.75ex]{CC0-small.png}}
\href{https://www.christophegouel.com}{Christophe Gouel}. Include materials from \emph{GAMS -- A User's Guide} (with permissions). \\
See \url{https://github.com/christophe-gouel/GAMS-Cheat-Sheet} for the sources.

Revision: 1.6.0, Date: \today{}.

For more information, see \href{https://www.gams.com/latest/docs/}{GAMS
  Documentation Center} and \href{https://www.youtube.com/user/GAMSLessons}{GAMS
  Lessons YouTube channel}.

\end{multicols}

\end{document}

%%% Local Variables:
%%% mode: LaTeX
%%% TeX-master: t
%%% End:
